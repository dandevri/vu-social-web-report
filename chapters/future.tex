\section{Future work}

Each of the parties is placed on a political spectrum (left, center left, centre, centre right, right). Quote a source.
There is probably an 'official' list for this.
Based on what they voted (maybe stemmentracker)? If they are left-wing, right-wing. How many zetels etc. this could be part of our weighting. We now say each party is equal in our graphs. But some parties are 'larger' in terms of ledennummers etc.

The list of topics should be larger.
We now manually made a list of topics based on summaries and manually scanning trough table of content of manifestos.
This could be automated in future work by having NLP scan table of contents from election manifestos, motions or determine the topic using classification based on voting guides (stemwijzers).

The network analysis focusses on political parties, but most party have one 'fractievoorzitter'.
To further analyse the network, we could also switch to individual user accounts of the fractievoorzitter's of the kamer and see based on whom they follow or their follower account what other politicians of the party have accounts on Mastodon.

The topics displayed in the voting guides are the topics parties mostly disagree on and are picked based on a larger list of topics. 
That doesn't necessary mean these topics are the most popular topics in society. 
For future research, we would like to cross-reference topics from the voting guides with topics discussed on social networks to see if topics that gain traction on social networking sites align with topics from the stemwijzer.