\section{Future work}

In future work, it will be important to consider several key factors to enhance the effectiveness and enrich the datasets used in this research which are described in more detail below.

Currently, this research uses a subset of metadata available for each political party. Mainly the party names and election results. To further cross-reference the Mastodon we could enrich the metadata for each party to incorporate political spectrum data (e.g. official grouping from ProDemos or how parties voted based on the StemmenTracker \footnote{https://home.stemmentracker.nl/}) and 'weigh' each party based on left-wing, neutral or right-wing. With this the research could incorporate a 'popularity weighting' of each party since our current visualization treat each party the same not taking into account the amount of seats in the house of representatives or number of members.

The (Network) analysis currently also focusses on political parties and thus instances or servers related to the political parties but analysation of individual user accounts of faction leaders is not yet performed. To further analyse the network and get an overview of connections analysyation of personal accounts need to be performed to plot further relations of the network based on who faction leaders follow or who are following them  (e.g. handshake lemma) \cite{handshake}.

Query words and list of election-related topics in this research are manually labelled. This can be further expanded by automating this task, for example using Natural Language Processing (NLP) to scan through the table of contents of election manifestos or determine topics automatically based on keywords in quotes from voting guides. This also further enhances the reliability of the data since topics displayed in voting guides are what parties mostly disagree on. This doesn't necessarily mean these topics are the most popular topics in society among voters. For future research, it would also be interesting to cross-reference topics from the voting guides with topics discussed on social networks to see if topics that gain traction on social networking sites align with topics from voting guides.