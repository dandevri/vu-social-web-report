\section{Methodology}

\subsection{Datasets}

To check, validate and cross-reference our sub-questions we complement this data with three additional data sources: 

\begin{itemize}
  \item \textbf{Institut Public de Sondage d'Opinion Secteur (IPSOS) exitpoll:} a market research company which, commisioned by the 'Nederlandse Omroep Stichting' \footnote{https://nos.nl/} (NOS; English: Dutch Broadcasting Foundation) publishes market research about the elections (e.g. which voters switch between parties, which municipilaties has switched the most between parties) \cite{nos}.
  \item \textbf{Government Open Data (overheid.nl)}: specifically the datasets from The Dutch Electoral Council \footnote{https://www.kiesraad.nl/} (Dutch: Kiesraad), the government body that is responsibly for counting of the votes and publishing the results \cite{kiesraad}.
  \item \textbf{ProDemos voting guide (stemwijzer)}: a voting guide called Stemwijzer \footnote{https://home.stemwijzer.nl/} with pre-defined topics. By answering 30 statements with agree, disagree or no opinion, voters can compare their positions with those of political parties. Many of these voting guides exist, ProDemos is most requested and partly funded by the dutch government \cite{prodemos}.
\end{itemize}