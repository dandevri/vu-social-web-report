\section{Related Work}

Social networking sites and other popular online platforms have been used as a means to express political opinions and show support and/or dissent for particular political parties or ideologies quite regularly during election periods since the rise of social media. Citizens are able, thanks to social media, to follow politicians, political commentators, and political consultants as a way to spread messages of endorsement and opposition through the sharing and posting of digital content; this phenomenon known as 'political advertising' took place even before the rise of social media and allowed for properly structured campaign advertisements to take place on digital platforms (https://blog.oup.com/2017/06/history-political-social-media/) As a matter of fact, as well as political opinions and views provided by individual users or voters, political parties use these platforms and the underlying channels of communication to run their political campaigns and thus, draw further attention for political debate and discourse. 

Since the officially recorded use of social media for political campaigns, with Barack Obama's electoral campaign run in 2008, until current times, social media has been one of the primary political ideology battlefields across the globe, optimally leveraging the growing number of social media users and consequently that of people eligible to vote. (https://journals.sagepub.com/doi/10.1177/20563051211063461) This incremental evolution saw a deeper and more analytical organization of campaigns and political discourse on social media platforms effectively making these platforms key players in the domain of digital political journalism. (Kreiss D. (2012). \textit{Taking our country back: The crafting of networked politics from Howard Dean to Barack Obama}. Oxford University Press.)

Consequently, online political discourse through digital platforms became 
