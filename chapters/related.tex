\section{Related Work}

Social networking sites and other popular online platforms have been used as a means to express political opinions and show support and/or dissent for particular political parties or ideologies quite regularly during election periods since the rise of social media. Citizens are able, thanks to social media, to follow politicians, political commentators, and political consultants as a way to spread messages of endorsement and opposition through the sharing and posting of digital content; this phenomenon known as 'political advertising' took place even before the rise of social media and consequently allowed for properly structured campaign advertisements to take place on digital platforms \cite{website}.
As a matter of fact, as well as political opinions and views provided by individual users or voters, political parties use these platforms and the underlying channels of communication to run their political campaigns and thus, draw further attention for political debate and discourse. 

Since the officially recorded use of social media for political campaigns, with Barack Obama's electoral campaign run in 2008, until current times, social media has been one of the primary political ideology battlefields across the globe, optimally leveraging the growing number of social media users and consequently that of people eligible to vote \cite{socialmediasociety} . 
This incremental evolution saw a deeper and more analytical organization of campaigns and political discourse on social media platforms effectively making these platforms key players in the domain of digital political journalism, verifying the power social media can have in shaping political journalism \cite{kreiss}. 

As the number of political parties using social media for their electoral campaigns grew over time, research on how people's election voting style and their political ideology representation on these platforms has been of interest in the field of social web. Beyond the digital electoral campaign runs, research shows that there have been multiple instances where social media explicitly influenced the outcome of the election itself. In the paper by Fujiwara, Muller and Schwarz on the effect of twitter on the US presidential elections of 2016 and 2020, the text analysis of primarily tweets, survey data and other references show that the Republican vote got lowered in 2020 compared to 2016 as a direct effect of Twitter's influence on people's voting \cite{socialmediaelections}.
Political elections constitute a social web phenomena of great research interest as they are known to cause a considerable uptick in social media interactions and overall usage. For instance, during the 2012 US presidential elections, 327,452 tweets per minute were hit at peak level \cite{slide}.

For this project, we aim to find out and evaluate if a similar influence from social media sites has been experienced on the other side of the ocean, focusing specifically on the recent 2023 Dutch elections. Local research such as that documented in a student thesis Our choice of social media site, Mastodon, was driven by a practical reason, given by the ease of relevant data extraction due to the freely accessible API. Most importantly, however, we wanted to assess the validity of this SNS, with respect to dutch election voting, considering its novelty and the recent uptick in its use amongst the dutch population. Its novelty further gave us insights and research directions: in fact, we aim to appraise Mastodon's dutch election voting representation in 2023 as opposed to the previous dutch election as a way to additionally evaluate the growth of this social media platform. 

Both traditional and digital (or even online and offline) social media have been at the forefront of social web research with respect to their capabilities of providing visibility for a given political cause. In a thesis project carried out by a student at the University of Twente, results show that social media has an explicit agenda-setting effect on traditional media and reports the main cause to be the amount of visibility that a certain political ideology or political party gains on these platforms \cite{socialmediavisibility}.
In this research, differently from the researches about election voting briefly discussed above, we opt for a network analysis approach in the data analysis phase to visualize the relationships amongst various political parties and other relevant entities as a means to infer their visibility and assess their representation. 

