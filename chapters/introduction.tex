\section{Introduction}

In order to investigate this social web related topic, this study aims to answer the research question:
\textbf{\textit{"To what extent is the relatively new Social Networking Site Mastodon representative of the election voting of the dutch population?"}}. To answer this research question in-depth, the following sub-questions were formulated:

\begin{itemize}
  \item \textbf{R1:} \textit{What's the distribution of political parties on the platform and do they align with the outcome of the election? }
  \item \textbf{R2:} \textit{What political topics are discussed in posts and are they representative of the election manifesto of political parties? }
  \item \textbf{R3:} \textit{Do the topics that are discussed on the platform align with popular voting guides?}
\end{itemize}

The sub-questions are relevant to the main research question as they provide a more detailed and specific understanding of the topic. For our research we use Mastodon as a Social Networking site (SNSen) as case study and main data source but this research can be further expanded to any social network if the platform has an API that exposes platform data and has the characteristics of a typical social network. To check, validate and cross-reference our sub-questions we complement this data with three additional data sources: 

\begin{itemize}
  \item \textbf{Institut Public de Sondage d'Opinion Secteur (IPSOS) exitpoll:} a market research company which, commisioned by the 'Nederlandse Omroep Stichting' \footnote{https://nos.nl/} (NOS; English: Dutch Broadcasting Foundation) publishes market research about the elections (e.g. which voters switch between parties, which municipilaties has switched the most between parties) \cite{nos}.
  \item \textbf{Government Open Data (overheid.nl)}: specifically the datasets from The Dutch Electoral Council \footnote{https://www.kiesraad.nl/} (Dutch: Kiesraad), the government body that is responsibly for counting of the votes and publishing the results \cite{kiesraad}.
  \item \textbf{ProDemos voting guide (stemwijzer)}: a voting guide called Stemwijzer \footnote{https://home.stemwijzer.nl/} with pre-defined topics. By answering 30 statements with agree, disagree or no opinion, voters can compare their positions with those of political parties. Many of these voting guides exist, ProDemos is most requested and partly funded by the dutch government \cite{prodemos}.
\end{itemize}
