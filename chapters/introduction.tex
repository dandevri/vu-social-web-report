\section{Introduction}

On November the 22th 2023 around 77.7\% (13,473,750 eligible voters casted 10,475,139 votes in total) of the Dutch Population went to a polling station in their muncipality to vote for their political party of choice for the Dutch House of Representatives \cite{kies}. Of the 26 parties that participated in the election, 15 parties received enough votes for a seat in the House of Representatives.

Prior to elections viewpoints and topics of particular parties are discussed on Social Networking Sites (SNS). E.g. users post their support (or anti-support) for a specific political party, discuss topics that are mentioned in parties election manifesto, and discuss candidates that are on the election list.

One of these relatively new and emerging Social Networking Sites is Mastodon \footnote{https://joinmastodon.org/} a self-hosted social network with microblogging features similar to X\footnote{https://twitter.com/} (formerly known as Twitter) which we use for this research. Analysing Mastodon is interesting for two main reasons. 1) Since it's release, especially, the last two years Mastodon has seen a massive surge in increase of users and activity (e.g. posts, interaction) on the Platform, from around 3.500.000 in october 2022 to 8.100.000 users in october 2023 \cite{analytics}. One main reason for this exponential growth is the acquisition of Twitter by Elon Musk \cite{musk} with many users from Twitter transitioning to Mastodon. 2) Elections for the Dutch house of representatives only occur every 4 years. When Mastodon was initially released the number of users and activity on the platform was relatively low compared to other SNS's. As mentioned before, the last two years the platform grew and we've even seen dutch political parties create Mastodon instances for their party members (e.g. Bij1 \footnote{https://social.bij1.org/about}, Piratenpartij \footnote{https://mastodon.social/@Piratenpartij@social.globalpirates.net}) which means Mastodon increasinly becomes more representative of the dutch voting population (eligible voters).

In order to investigate this social web related topic, this study aims to answer the research question:
\textbf{\textit{"To what extent is the relatively new Social Networking Site Mastodon representative of the election voting of the dutch population?"}}. To answer this research question in-depth, the following sub-questions were formulated:

\begin{itemize}
  \item \textbf{R1:} \textit{What's the distribution of political parties on the platform and do they align with the outcome of the election? }
  \item \textbf{R2:} \textit{What political topics are discussed in posts and are they representative of the election manifesto of political parties? }
  \item \textbf{R3:} \textit{Do the topics that are discussed on the platform align with popular voting guides?}
\end{itemize}

The sub-questions are relevant to the main research question as they provide a more detailed and specific understanding of the topic. For our research we use Mastodon as a Social Networking site (SNSen) as case study and main data source but this research can be further expanded to any social network if the platform has an API that exposes platform data and has the characteristics of a typical social network. To check, validate and cross-reference our sub-questions we complement this data with three additional data sources: 

\begin{itemize}
  \item \textbf{Institut Public de Sondage d'Opinion Secteur (IPSOS) exitpoll:} a market research company which, commisioned by the 'Nederlandse Omroep Stichting' \footnote{https://nos.nl/} (NOS; English: Dutch Broadcasting Foundation) publishes market research about the elections (e.g. which voters switch between parties, which municipilaties has switched the most between parties) \cite{nos}.
  \item \textbf{Government Open Data (overheid.nl)}: specifically the datasets from The Dutch Electoral Council \footnote{https://www.kiesraad.nl/} (Dutch: Kiesraad), the government body that is responsibly for counting of the votes and publishing the results \cite{kiesraad}.
  \item \textbf{ProDemos voting guide (stemwijzer)}: a voting guide called Stemwijzer \footnote{https://home.stemwijzer.nl/} with pre-defined topics. By answering 30 statements with agree, disagree or no opinion, voters can compare their positions with those of political parties. Many of these voting guides exist, ProDemos is most requested and partly funded by the dutch government \cite{prodemos}.
\end{itemize}
