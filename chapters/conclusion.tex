\section{Conclusion}


The research presented in this paper delved into the dynamics of political activity on the Mastodon platform, focusing on the Dutch House of Representatives elections. The results, analyzed through three research questions, provided valuable insights into the adoption and engagement of political parties on this decentralized social network.

Firstly, the overall activity surrounding the Dutch elections on Mastodon exhibited a significant increase, particularly in the most recent election year, 2023 contrasted with minimal activity observed in the preceding elections of 2017 and 2021. The examination of the activity of political parties revealed that while a few parties dominated the discussion, the nature of their engagement varied. The Socialist Party (SP) emerged as a notable outlier, prominently featured in discussions, possibly influenced by the prevalence of their initials in the Dutch language. Interestingly, parties like the Piratenpartij demonstrated active engagement, with their own Mastodon instance, while others, like the GL-PvdA party, garnered attention despite not being directly active on the platform.

Furthermore, the analysis of election-related topics and query words highlighted the prevalence of certain themes, such as climate and the economy, predominantly associated with left-leaning parties. However, no distinct differences in discourse were observed between parties, emphasizing the even spread of discussions across various topics when parties with activity on the platform were mentioned.

Finally, the exploration of party accounts and servers revealed that a limited number of parties actively participated on Mastodon. Notably, some parties established their own instances, reflecting a commitment to decentralized structures and specialized platforms. The Piratenpartij, with its focus on internet laws and net neutrality, and Bij1, advocating for decentralized structures, exemplified this trend.

In summary, Mastodon's role in political discourse around Dutch elections has evolved, with increased activity and nuanced engagement by political parties. The findings contribute to our understanding of the dynamics of political discussions in decentralized social networks, shedding light on the platforms and strategies adopted by parties to navigate the digital social media networking landscape. With this work, we invite researchers, journalists, and practitioners alike to further investigate Mastodon in relation to the Dutch House of Representatives elections and explore any other new and upcoming Social Networking Site using similar methodology.