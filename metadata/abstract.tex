\begin{abstract}
Prior to elections viewpoints of parties and individual users  are often expressed on Social Networking Sites (SNS) since expressing opinition is one of the core characteristics of social media. Recently, (November 2023) elections for the Dutch House of Representatives took place. This study focuses on on a data-driven social web approach aiming to answer the research question: \textit{to what extent is the relatively new SNS Mastodon representative of the election voting of the dutch population?} In the analysis this research processed data from several election-related resources such as political parties and topics from election manifestos and are cross-referenced and statistically analysed with Mastodon instance-wide (e.g. servers) and Streaming API (e.g. toots and mentions). The results indicated that there is a significant increase in activity related to the dutch election on the Mastodon platform and political parties and candidates are increasingly creating user accounts and instances. Privacy and ethical considerations when accessing Mastodon platform data are discussed and in the future work section, the study addresses several further enhancement that can be performed to automate more of the processing methods and to further expand the election data sources used.

\keywords{Social Web  \and Social Network \and Network Analysis \and Mastodon \and Dutch Elections \and Political Parties \and User-generated Content.}
\end{abstract}